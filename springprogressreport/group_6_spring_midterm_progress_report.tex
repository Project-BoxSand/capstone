\documentclass[onecolumn, draftclsnofoot,10pt, compsoc]{IEEEtran}
\usepackage{graphicx}
\usepackage{url}
\usepackage{setspace}
\usepackage{pdfpages}
\usepackage{lscape}
\usepackage[noadjust]{cite}  
\usepackage{multicol}
\usepackage{geometry}
\geometry{textheight=9.5in, textwidth=7in}

% 1. Fill in these details
\def \CapstoneTeamName{		Project BoxSand}
\def \CapstoneTeamNumber{		6}
\def \GroupMemberOne{			Max Moulds}
\def \GroupMemberTwo{			Sam Morey}
\def \GroupMemberThree{			Anya Lehman}
\def \CapstoneProjectName{		Project BoxSand}
\def \CapstoneSponsorCompany{	Oregon State Physics Department}
\def \CapstoneSponsorPerson{		Dr. Kenneth Walsh}

% 2. Uncomment the appropriate line below so that the document type works
\def \DocType{		Progress Report
				%Requirements Document
				%Technology Review
				%Design Document
				%Progress Report
				}
			
\newcommand{\NameSigPair}[1]{\par
\makebox[2.75in][r]{#1} \hfil 	\makebox[3.25in]{\makebox[2.25in]{\hrulefill} \hfill		\makebox[.75in]{\hrulefill}}
\par\vspace{-12pt} \textit{\tiny\noindent
\makebox[2.75in]{} \hfil		\makebox[3.25in]{\makebox[2.25in][r]{Signature} \hfill	\makebox[.75in][r]{Date}}}}
% 3. If the document is not to be signed, uncomment the RENEWcommand below
%\renewcommand{\NameSigPair}[1]{#1}


%%%%%%%%%%%%%%%%%%%%%%%%%%%%%%%%%%%%%%%
\begin{document}
\begin{titlepage}
    \pagenumbering{gobble}
    \begin{singlespace}
        \hfill 
        % 4. If you have a logo, use this includegraphics command to put it on the coversheet.
        %\includegraphics[height=4cm]{CompanyLogo}   
        \par\vspace{.2in}
        \centering
        \scshape{
            \huge CS Capstone \DocType \par
            {\large\today}\par
            \vspace{.5in}
            \textbf{\Huge\CapstoneProjectName}\par
            \vfill
            {\large Prepared for}\par
            \Huge \CapstoneSponsorCompany\par
            \vspace{5pt}
            {\Large\NameSigPair{\CapstoneSponsorPerson}\par}
            {\large Prepared by }\par
            Group\CapstoneTeamNumber\par
            % 5. comment out the line below this one if you do not wish to name your team
            \CapstoneTeamName\par 
            \vspace{5pt}
            {\Large
                \NameSigPair{\GroupMemberOne}\par
                \NameSigPair{\GroupMemberTwo}\par
                \NameSigPair{\GroupMemberThree}\par
            }
            \vspace{20pt}
        }
\begin{abstract}
The purpose of this document is to serve as a recap of the progress completed by Team BoxSand on the BoxSand project thus far. The document will contain an introduction to the project, a summary of the current state of the project, what is left to do in the project, problems the development team has run into, a retrospective covering the last 10 weeks, a brief evaluation of the members of the team, and a conclusion. 
\end{abstract}
    \end{singlespace}
\end{titlepage}
\newpage
\pagenumbering{arabic}

\tableofcontents
% 7. uncomment this (if applicable). Consider adding a page break.
%\listoffigures
%\listoftables

\clearpage
% 8. now you write!
\singlespacing

\section{Introduction}
At its core BoxSand is driven by a need to create an all-in-one open source learning website that is free and available for use by both students and instructors. Currently, BoxSand allows its users to access educational resources that already exist but needs a design overhaul to help usher in a new wave of content and contribution while still providing the same level of easy access. The current BoxSand links to dozens of other websites for material, and uses a non-free service as a homework system. The long term goal is to have a modular website where instructors can create a course in any subject and contribute and edit content. The goal for this development team is to set up an environment where this is possible. The development team will design, document, and begin implementation of that framework. This includes the tracking of student interaction with the site while using the OpenStax suite of technologies and adding a homework system. Completing this task may require substantial changes to the underlying codebase of the project, with special attention towards decreasing code complexity, maximizing maintainability to assist future development and fostering new content contribution.

\section{Current State}
The project is in the final phases of this development cycle. We are applying the style elements and updating licensing and other bannering. Accounts is almost fully completed. Work on tutor server has continued, most around importing and deployments. The status of exercises is volatile, since it is one of the odd sources being that is is node where all the others we are primarily working on are Rails apps this one is particularly stubborn. JS and image assets are also nearly complete if not complete. This phase also involves final documentation of deployment processes and transitions for next years Capstone team. 

In the first progress report we dealt with github and basic toolbelting issues along with developing documentation for the project goals and expectations. We also dealt with initial deployment issues and HTTP/S, SSL, DNS, and some odd hosting and environment issues related to new provider and upstream updates. 
In the second progress report we discussed our solutions to some of those problems and introduced new issues with resource limits on some components of the site, database confusion between components of the app, and general understanding of intra-app routing and flow of information. Some problems that arose in this timeframe that are outstanding include redis intermittent connection issues that seem to be related to how rails ties in with redis and how it is called by the app itself. Also, updates during this timeframe took a considerable amount of development time from progress with known issues to resolve issues based in those updates. For the most part, every one of those issues was resolved. We also completed the icon pack and some of the style ideas for the site during this timeframe. 

For this report some past issues have been resolved and some that were resolved but left partially implemented have been completed. This includes the style updates, icon packs, and updating terms and some of the licensing. 


\section{Outstanding Work}
Accounts needs testing scripts or some sort of verification procedure for Learn@OSU and Canvas. Also, we need to finalize the document explaining it use for next years development team. Tutor server needs to properly configured to allow Redis to accept import data. This has been spotty and is discussed in the Problems section. 

Where deployments are concerned, a final docker image needs to be created along with an updated install/deployment process. It would also be beneficial to update the dependencies since many gems and packages have been updated this year mostly due to security concerns. This may take some time, considering the largest part of the project seems deeply dependent on a specific version of Ruby that is being deprecated. 

Some major past issues that have been resolved were with the accounts portion of the project, relating to authorization of apps using Oauth2. These issues have been resolved and documented more extensively in this term which has helped progression in other areas. Another past issue involved Postgres and user accounts, this has also been resolved. Current issues surrounding redis and imports are our focus and have been the source of the most time lost. This issue is complex because it requires all aspects of the site to be functioning for even some of the most basic testing, including js, exercises, accounts, and tutor server itself, and while it does not directly inhibit development on other areas of the site, it makes progress harder. 

In addition to ongoing development, time has been committed to wrapping up the project in order to meet the Capstone requirements. The team has dedicated some time to interfacing with a group of students that have self-identified as potential team members for the Capstone project for next year. This has included 3 in-person meetings and multiple online meetings to discuss interdependencies between the two projects and to gradually onboard the new members. Some discussion has been extremely helpful in review of our own progress and problems. We have also begun testing deployments on what will be our final production server environment. 


\section{Problems}
One of the biggest problems so far has been importing and Redis. The errors occurring due to this problem are very minimally descriptive and extremely hard to debug, leading to much frustration. Since this aspect is very important to the client and also to the site operation it is very important. The Redis client and server have been tested with other apps and is working which points to some configuration or improper use for our deployment of Redis by the tutor server component. This error may also be related to the time cop issue that is plaguing exercises. That issue is more of a mystery since it is newer and only surfaced this past week. It is believed to be an issue with a gem or an updated syntax that is not reflected in the source yet. 

Looking back at Winter quarter's issue list it was noticed that it was about a month for upstream fixes to find their way down to our components. This may not be fast enough for some issues and we may have to revert to older version for a demo for the Expo. 

Another issue is lack of full story testing. Since we have had considerable issue connecting the app completely if has been hard to fully test edge features and mainly tracking. This data gathered by tracking is a listed requirement and is still being addressed in our planning for the rest of the development cycle but chances are that it will not meet final deadline and therefore must be considered as an issue beyond lack of testing. A better estimate to when this feature can be fully tested is going to be hard to derive and while work will still continue. the team should address this with the client. 

One problem that is not really a software or hardware issue but instead a forward looking one involves the proposed team for the next iteration of this Capstone project. That team has expressed strong interest and desire to use socket.io and node exclusively for the implementation of that app. This will further enlarge our dependencies and may cause more problems that it would solve by using node and socket.io. Some research should be done before any commitments are done to ensure the viability of that change of process and to see if there is any immediate issues that would hamper overall site function. 

Since a new team is being readied now, it has given the project a chance to reflect on processes and documentation created in the beginning and discern the effectiveness of that work. It has been noted that updates to that past work will be needed continually in order to keep the work relevant and applicable and this may not be feasible given the availability of the team. Some documentation and processes must be kept up to date. A small portion of current work should be devoted to ensuring past work stays relevant and this portion should be more than is currently allotted. As a comment on this issue, I fear that there is too much to maintain without a more permanent and larger development team. This issue has been discussed with the client and solutions are being identified. 

Preparing for Expo has also been a problem. While not a critical one nor one that has help up progress in other areas, it has been a source of angst. For one, it is eight hours of demonstration of something that is not going to be fully functional at that time. It will have a demo version but it will not be what the client will be receiving after this development cycle. This has placed some extra stress on the team to deliver working results over implementing lasting and quality solutions. Since it was recognized early, a plan to minimize this effect has been put in place, but it still had an effect. 

 \section{Conclusion}
 Progress is continuing. While the project is technically behind the schedule outlayed during milestoning and planning we had also planned a roadmap for this pace of progress. It involved a summer cycle focusing on transition and final completion of any latent issues. We are confident that the majority of client requirements will be met in that timeframe. More work will continue on app feature development but most work will be focused on existing feature support and documentation.
 
 

\end{document}
