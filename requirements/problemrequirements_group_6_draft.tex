\documentclass[onecolumn, draftclsnofoot,10pt, compsoc]{IEEEtran}
\usepackage{graphicx}
\usepackage{url}
\usepackage{setspace}


\usepackage{geometry}
\geometry{textheight=9.5in, textwidth=7in}

% 1. Fill in these details
\def \CapstoneTeamName{		Project BoxSand}
\def \CapstoneTeamNumber{		6}
\def \GroupMemberOne{			Max Moulds}
\def \GroupMemberTwo{			Sam Morey}
\def \GroupMemberThree{			Anya Lehman}
\def \CapstoneProjectName{		Project BoxSand}
\def \CapstoneSponsorCompany{	Oregon State Physics Department}
\def \CapstoneSponsorPerson{		Dr. Kenneth Walsh}

% 2. Uncomment the appropriate line below so that the document type works
\def \DocType{		Problem Requirements
				%Requirements Document
				%Technology Review
				%Design Document
				%Progress Report
				}
			
\newcommand{\NameSigPair}[1]{\par
\makebox[2.75in][r]{#1} \hfil 	\makebox[3.25in]{\makebox[2.25in]{\hrulefill} \hfill		\makebox[.75in]{\hrulefill}}
\par\vspace{-12pt} \textit{\tiny\noindent
\makebox[2.75in]{} \hfil		\makebox[3.25in]{\makebox[2.25in][r]{Signature} \hfill	\makebox[.75in][r]{Date}}}}
% 3. If the document is not to be signed, uncomment the RENEWcommand below
%\renewcommand{\NameSigPair}[1]{#1}

%%%%%%%%%%%%%%%%%%%%%%%%%%%%%%%%%%%%%%%
\begin{document}
\begin{titlepage}
    \pagenumbering{gobble}
    \begin{singlespace}
        \hfill 
        % 4. If you have a logo, use this includegraphics command to put it on the coversheet.
        %\includegraphics[height=4cm]{CompanyLogo}   
        \par\vspace{.2in}
        \centering
        \scshape{
            \huge CS Capstone \DocType \par
            {\large\today}\par
            \vspace{.5in}
            \textbf{\Huge\CapstoneProjectName}\par
            \vfill
            {\large Prepared for}\par
            \Huge \CapstoneSponsorCompany\par
            \vspace{5pt}
            {\Large\NameSigPair{\CapstoneSponsorPerson}\par}
            {\large Prepared by }\par
            Group\CapstoneTeamNumber\par
            % 5. comment out the line below this one if you do not wish to name your team
            \CapstoneTeamName\par 
            \vspace{5pt}
            {\Large
                \NameSigPair{\GroupMemberOne}\par
                \NameSigPair{\GroupMemberTwo}\par
                \NameSigPair{\GroupMemberThree}\par
            }
            \vspace{20pt}
        }

    \end{singlespace}
\end{titlepage}
\newpage
\pagenumbering{arabic}
\tableofcontents
% 7. uncomment this (if applicable). Consider adding a page break.
%\listoffigures
%\listoftables
\clearpage

% 8. now you write!

\section{Introduction}
This project aims to develop an open source Learning Module System that will be used by students to learn course material and complete online homework. The project, titled BoxSand, provides Open Educational Resources with a primary goal to provide students and instructors with an all-in-one online learning environment. This includes links to videos, practice problems, open source textbooks, educational websites, simulations, and more. The main goal of BoxSand is to improve student performance by providing access to free and open source resources all in one location. It aims to engage students and provide them with feedback while using the student's site interaction and engagement to track success and improve content.

\subsection{Purpose}
Since this is the first large-scale overhaul of the BoxSand project, this development team and this cycle should mainly focus on developing long-term and overarching development goals and procedures for future project development. Additionally, this development team will also create an initial proof of concept for site functionality demonstration and future feature integration


\subsection{Scope}
This development team will produce the BoxSand OpenStax site. At its core BoxSand is driven by a need to create an all-in-one open source learning website that is free and available for use by both students and instructors. The long term goal is to have a modular website where instructors can create a course in any subject and contribute and edit content. The goal for this development team is to set up an environment where this is possible. The development team will design, document, and begin implementation of that framework. This includes the tracking of student interaction with the site while using the OpenStax suite of technologies and adding a homework system. This first iteration of the project will be successfully completed when the following criteria are met:

\begin{enumerate}
\item[1] The website that will be developed must provide access to the OpenStax Physics textbook within the site itself. 
\item[2] Provide a homework system within the site that allows an instructor of a course to provide questions with answers, Assign a value to the question and, assign a group of questions or a single question as an assignment to a course.
\item[3] Instructor must also be able to assign reading homework from the textbook for students within a course
\item[4] An instructor must be able to generate a downloadable gradebook of student scores
\end{enumerate}


\section{Overall Description}

\subsection{Product Perspective}
BoxSand's end goal could most easily be compared to a mix between a MOOC, such as coursera, and a more general site for learning such as Khan Academy. The BoxSand will use resources from an already existing open source learning site OpenStax.

\subsubsection{System Interfaces}

\subsubsection{User Interfaces}

\begin{enumerate}
\item The BoxSand website will have the following pages:
\begin{enumerate}
\item Landing page: Where a non signed in user will be directed upon arriving at the BoxSand website. This page will include a small about section and options to either login through ONID or visit as a guest. 
\item User Dashboard: This page will show a signed in user what they have accomplished in their course so far. It will include a small section for an optional profile picture and description. Below that it will show what the user has completed in the course by breaking their course into modules.
\item Content pages: These pages will vary depending on what content is on the page. They will have a variation of: images, videos, textbook info, and homework depending on what the page will be showing. In general these pages will have a central area for the content and a section one the left hand side to show where the user is in the current module. 
\end{enumerate}
\item The goal is to make the students life as easy as possible and reduce their total number of clicks as much as possible. 
\item To strive for this keep navigation menus simple and trimmed down. 
\item Use clearly defined things such as drop downs. 
\item A student should know how to navigate between the learning modules and account within an hour of exploring the site. 
\end{enumerate}

\subsubsection{Hardware Interfaces}

\subsubsection{Software Interfaces}
Name: OpenStax Tutor 
Mnemonic: N/A
Specification Number:
Version Number:
Source: 

\subsubsection{Communication Interfaces}

\subsubsection{Memory Constraints}

\subsubsection{Operations}

\begin{enumerate}
\item The user will be able to :
\begin{enumerate}
\item Navigate the website through menus
\item Edit their personal profile
\item Answer homework questions
\item View simulations/images/videos
\end{enumerate}
\item Interactive operations:
\begin{enumerate}
\item Editing profile
\item Answering questions
\item Unattended operations:
\item Reading material
\item Watching videos
\end{enumerate}
\end{enumerate}

\subsubsection{Site adaptation requirements}

\subsection{Product Functions}

\begin{enumerate}
\item The end goal of this iteration of BoxSand is to have the following:
\begin{enumerate}
\item Provide access to the OpenStax text book
\item A homework system that allows the instructor to add questions to the course.
\item Ability for the instructor to assign homework
\item Ability for instructor to view student's grades 
\item Student Dashboard for viewing their progress
\end{enumerate}
\end{enumerate}

\subsection{User Characteristics}
There are two main types of user for the BoxSand website. One is the instructor, who will usually be a master in their respective field most often having completed a masters degree or above (ages vary). The other user is the student, which will typically consist of an undergraduate student with a typical age range from 18-24. 

\subsection{Constraints}

\begin{enumerate}
\item Regulatory policies:
\item Hardware limitations: 
\item Interfaces to other applications: Dependent on how willing OpenStax is to working with us.
\item Parallel operation:
\item Audit functions:
\item Control functions:
\item Higher-order language requirements:
\item Signal handshake protocols:
\item Reliability requirements:
\item Criticality of the application:
\item Safety and Security Concerns: Since we will be using OSU's CAS login system, we will be partially dependent on their security. 
\end{enumerate}

\subsection{Assumptions and Dependencies}

\section{Specific Requirements}

\subsection{External Interfaces}

To achieve our development goals we will need to create a website that provides four core functions. One, it will integrate the OpenStax physics textbook within the BoxSand site. Two, we will also have a homework system integrated with OpenStax QuadBase questions. This will be implemented in the same way as the OpenStax textbook in that the students will not have to leave the BoxSand site in order to work on their homework. Three, we will migrate the information from the existing BoxSand site. Four, the instructor of the course will be able to monitor the student's relevant progress and site interaction. Such data includes time spent in text book, time spent watching videos, time spent on homework, and homework score. This information will be used by the instructor to study student performance.


\section{Design Constraints}


\section{Supporting Information}

\subsection{Appendixes}

\subsubsection{Appendix B - Gantt Chart}
See the Project BoxSand Roadmap document.



\end{document}
