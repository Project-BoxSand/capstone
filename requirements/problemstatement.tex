
\documentclass[10pt,journal,compsoc,draftclsnofoot,onecolumn]{IEEEtran}

\ifCLASSOPTIONcompsoc
  \usepackage[nocompress]{cite}
\else
  % normal IEEE
  \usepackage{cite}
\fi

\ifCLASSINFOpdf

\else

\fi

\newcommand\MYhyperrefoptions{bookmarks=true,bookmarksnumbered=true,
pdfpagemode={UseOutlines},plainpages=false,pdfpagelabels=true,
colorlinks=true,linkcolor={black},citecolor={black},urlcolor={black},
pdftitle={Project BoxSand Problem Statement},%<!CHANGE!
pdfsubject={Typesetting},%<!CHANGE!
pdfauthor={Max Moulds},%<!CHANGE!
pdfkeywords={Computer Society, IEEEtran, journal, LaTeX, paper,
             template}}%<^!CHANGE!

\hyphenation{op-tical net-works semi-conduc-tor}


\begin{document}

\title{Problem Statement for Project BoxSand}


\author{Max Moulds
\IEEEcompsocitemizethanks{\IEEEcompsocthanksitem M. Moulds something aobut something why?\protect\\
E-mail: mouldsm@oregonstate.edu}
\thanks{Manuscript received April 19, 2005; revised August 26, 2015.}}


\markboth{Project BoxSand Problem Statement - Draft - October 9, 2017}%
{Shell \MakeLowercase{\textit{et al.}}: The Problem Statement (draft) for Project BoxSand}

\IEEEtitleabstractindextext{%
\begin{abstract}
Project BoxSand is an online learning environment that provides Physics students at Oregon State University with access to video, practice problems, open source textbooks educational websites, simulations and more. The project provides students with a single place to complete and study the subject while also providing instructors with valuable data on student interaction. The main goal of the project is to improve student performance by providing access to free and open source resources while using student site interaction and engagement to improve Project BoxSand's content. 
\end{abstract}


\begin{IEEEkeywords}
BoxSand, OpenStax, Open Source Learning Management Systems, Open Educational Resources, Open Textbooks.
\end{IEEEkeywords}}


% make the title area
\maketitle


\IEEEdisplaynontitleabstractindextext

\IEEEpeerreviewmaketitle

\pagebreak

\ifCLASSOPTIONcompsoc
\IEEEraisesectionheading{\section{Introduction}\label{sec:introduction}}
\else
\section{Introduction}
\label{sec:introduction}
\fi

\IEEEPARstart{C}{lassically} a college student must pay for tuition and books. As more resources are available in a digital format, students are required to gain access to more and more resources in order to complete course requirements. The aim of this project is to bring the whole of course resource requirements to one place, BoxSand. This would include the textbooks, questions, grading mechanisms and others. Bringing these features to one place also relieves the student from having to pay the extraneous cost from these online resources such as MasterPhysics or PeerCeptiv. In the long run, BoxSand will save students money while providing an improved learning environment. 

\hfill mm
 
\hfill October 9, 2017

\section{Learning Module System}
Currently the BoxSand website supports simple and marginally interactive resources. One future goal of the project is to extend functionality to incorporate the open learning management system OpenStax that is currently being actively developed. This would include the OpenStax open textbook resources and the QuadBase question bank. Additionally, OpenStax would integrate with existing learning management systems to help ease the transition from those systems to Project BoxSand. New content is important to the project's overall goals but existing content must also be transitioned from the old system to the new. 

\subsection{Features}
Learning modules for the site must be able to track student interaction like the existing site does while allowing a due date to be set for individual lessons. An example of the types of learning modules would include: reading sections, videos, or interactive simulations. These modules also need to be generic enough to work with future content. 

\section{Contribute Questions}
As the project grows, the subject of a learning module should be able to be changed by an instructor. This requires the project to adequately document how a course is built. OpenStax allows course material in many different subjects so maintaining a general solution to content addition may be too complicated but if the underlying system is well described, the project would be easily extended to other courses. Specifically for the physics course, a question bank supporting the common types of questions, multiple choice, free response, and others are similar to other future course additions like mathematics and engineering courses. Some new question types should also be considered when designing and documenting the overhaul. Features such as simulations manipulations, video driven experiences and other types should be enumerated for future implementation. 

\section{Student Dashboard}
The BoxSand site provides no overview for a student to track their progress. The site also lacks an easy way for students to converse and provide peer to peer assistance. The student dashboard feature would enable future improvements along those lines. Enhancements such as group chats, virtual whiteboards, peer grading and other forms of interaction can be facilitated from this framework. While implementing many of these features may be outside of the capabilities of this development process it should be something that is considered during the design phase and planned for accordingly. 

\section{Instructor Dashboard}
Project BoxSand currently is built around a content management system provided by Drupal. It is very good at providing easy lower-level customization for content providers such as instructors and teacher assistants. As the project grows the use of Drupal may need to be reevaluated with special consideration toward future maintainability and integrity. While it may be possible to create the feature set requested in this project it may impose large limitations on ease of maintenance and scalability. It is therefor important for this project to implement a fairly robust instructor dashboard. This feature must support class associations, listing enrollments, progress, and content associated with the class, along with providing a good place for instructors to add new content. 

\section{Export Student Performance Data}
The current site provides research data for Dr. Walsh, the project director. This should be built upon and well documented. The existing logging and data collection system is not as robust as it should be. Future enhancements include increased granularity, improved video and media interaction information, and better performance generating the reports documenting the collected data. 

\section{Async Sync and the Future}
The future for the project is bright. Some future ideas for improvement involve the gamification of course material, a more advanced adaptive learning algorithm, and improved student interaction. Mainly, Async Sync is the placeholder for the ultimate vision for the project - a truly adaptive learning management system. This adaptive system would adjust to individual students aptitude and would use a students past performance to craft more appropriate questions. This can be further applied to peer grouping. By studying the progress of groups of students in a class the future BoxSand would allow students in similar points of the course to contribute to each other's success. Features in this subset include peer to peer interaction and question or learning module scaling. When Async Sync is paired with the open, free, and easily contributed content system of BoxSand and OpenStax there are few, if any competitors that could provide the same level of learning experience.


\section{Conclusion}
Project BoxSand sets many ambitious goals and while it would be nice to achieve all of them, and planning should mirror that desire, it must be noted that some features are more imperative to the BoxSand site than others. The future is bright for the project and hopefully many of the features discussed will make it to the BoxSand site. 

\appendices
\section{Just Holding Place}
Appendix one text goes here.

% you can choose not to have a title for an appendix
% if you want by leaving the argument blank
\section{}
Appendix two text goes here.


% use section* for acknowledgment
\ifCLASSOPTIONcompsoc
  % The Computer Society usually uses the plural form
  \section*{Acknowledgments}
\else
  % regular IEEE prefers the singular form
  \section*{Acknowledgment}
\fi


The authors would like to thank...


% Can use something like this to put references on a page
% by themselves when using endfloat and the captionsoff option.
\ifCLASSOPTIONcaptionsoff
  \newpage
\fi


\begin{thebibliography}{1}

\bibitem{IEEEhowto:kopka}
H.~Kopka and P.~W. Daly, \emph{A Guide to {\LaTeX}}, 3rd~ed.\hskip 1em plus
  0.5em minus 0.4em\relax Harlow, England: Addison-Wesley, 1999.

\end{thebibliography}


\begin{IEEEbiography}{Max Moulds}
Biography text here. If I will have a photo, but probably not. 
\end{IEEEbiography}

% if you will not have a photo at all:
\begin{IEEEbiographynophoto}{John Doe}
Biography text here.
\end{IEEEbiographynophoto}


\begin{IEEEbiographynophoto}{Jane Doe}
Biography text here.
\end{IEEEbiographynophoto}

\end{document}


